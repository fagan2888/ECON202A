\documentclass[11pt]{article}

    \usepackage[breakable]{tcolorbox}
    \usepackage{parskip} % Stop auto-indenting (to mimic markdown behaviour)
    
    \usepackage{iftex}
    \ifPDFTeX
    	\usepackage[T1]{fontenc}
    	\usepackage{mathpazo}
    \else
    	\usepackage{fontspec}
    \fi

    % Basic figure setup, for now with no caption control since it's done
    % automatically by Pandoc (which extracts ![](path) syntax from Markdown).
    \usepackage{graphicx}
    % Maintain compatibility with old templates. Remove in nbconvert 6.0
    \let\Oldincludegraphics\includegraphics
    % Ensure that by default, figures have no caption (until we provide a
    % proper Figure object with a Caption API and a way to capture that
    % in the conversion process - todo).
    \usepackage{caption}
    \DeclareCaptionFormat{nocaption}{}
    \captionsetup{format=nocaption,aboveskip=0pt,belowskip=0pt}

    \usepackage[Export]{adjustbox} % Used to constrain images to a maximum size
    \adjustboxset{max size={0.9\linewidth}{0.9\paperheight}}
    \usepackage{float}
    \floatplacement{figure}{H} % forces figures to be placed at the correct location
    \usepackage{xcolor} % Allow colors to be defined
    \usepackage{enumerate} % Needed for markdown enumerations to work
    \usepackage{geometry} % Used to adjust the document margins
    \usepackage{amsmath} % Equations
    \usepackage{amssymb} % Equations
    \usepackage{textcomp} % defines textquotesingle
    % Hack from http://tex.stackexchange.com/a/47451/13684:
    \AtBeginDocument{%
        \def\PYZsq{\textquotesingle}% Upright quotes in Pygmentized code
    }
    \usepackage{upquote} % Upright quotes for verbatim code
    \usepackage{eurosym} % defines \euro
    \usepackage[mathletters]{ucs} % Extended unicode (utf-8) support
    \usepackage{fancyvrb} % verbatim replacement that allows latex
    \usepackage{grffile} % extends the file name processing of package graphics 
                         % to support a larger range
    \makeatletter % fix for grffile with XeLaTeX
    \def\Gread@@xetex#1{%
      \IfFileExists{"\Gin@base".bb}%
      {\Gread@eps{\Gin@base.bb}}%
      {\Gread@@xetex@aux#1}%
    }
    \makeatother

    % The hyperref package gives us a pdf with properly built
    % internal navigation ('pdf bookmarks' for the table of contents,
    % internal cross-reference links, web links for URLs, etc.)
    \usepackage{hyperref}
    % The default LaTeX title has an obnoxious amount of whitespace. By default,
    % titling removes some of it. It also provides customization options.
    \usepackage{titling}
    \usepackage{longtable} % longtable support required by pandoc >1.10
    \usepackage{booktabs}  % table support for pandoc > 1.12.2
    \usepackage[inline]{enumitem} % IRkernel/repr support (it uses the enumerate* environment)
    \usepackage[normalem]{ulem} % ulem is needed to support strikethroughs (\sout)
                                % normalem makes italics be italics, not underlines
    \usepackage{mathrsfs}
    

    
    % Colors for the hyperref package
    \definecolor{urlcolor}{rgb}{0,.145,.698}
    \definecolor{linkcolor}{rgb}{.71,0.21,0.01}
    \definecolor{citecolor}{rgb}{.12,.54,.11}

    % ANSI colors
    \definecolor{ansi-black}{HTML}{3E424D}
    \definecolor{ansi-black-intense}{HTML}{282C36}
    \definecolor{ansi-red}{HTML}{E75C58}
    \definecolor{ansi-red-intense}{HTML}{B22B31}
    \definecolor{ansi-green}{HTML}{00A250}
    \definecolor{ansi-green-intense}{HTML}{007427}
    \definecolor{ansi-yellow}{HTML}{DDB62B}
    \definecolor{ansi-yellow-intense}{HTML}{B27D12}
    \definecolor{ansi-blue}{HTML}{208FFB}
    \definecolor{ansi-blue-intense}{HTML}{0065CA}
    \definecolor{ansi-magenta}{HTML}{D160C4}
    \definecolor{ansi-magenta-intense}{HTML}{A03196}
    \definecolor{ansi-cyan}{HTML}{60C6C8}
    \definecolor{ansi-cyan-intense}{HTML}{258F8F}
    \definecolor{ansi-white}{HTML}{C5C1B4}
    \definecolor{ansi-white-intense}{HTML}{A1A6B2}
    \definecolor{ansi-default-inverse-fg}{HTML}{FFFFFF}
    \definecolor{ansi-default-inverse-bg}{HTML}{000000}

    % commands and environments needed by pandoc snippets
    % extracted from the output of `pandoc -s`
    \providecommand{\tightlist}{%
      \setlength{\itemsep}{0pt}\setlength{\parskip}{0pt}}
    \DefineVerbatimEnvironment{Highlighting}{Verbatim}{commandchars=\\\{\}}
    % Add ',fontsize=\small' for more characters per line
    \newenvironment{Shaded}{}{}
    \newcommand{\KeywordTok}[1]{\textcolor[rgb]{0.00,0.44,0.13}{\textbf{{#1}}}}
    \newcommand{\DataTypeTok}[1]{\textcolor[rgb]{0.56,0.13,0.00}{{#1}}}
    \newcommand{\DecValTok}[1]{\textcolor[rgb]{0.25,0.63,0.44}{{#1}}}
    \newcommand{\BaseNTok}[1]{\textcolor[rgb]{0.25,0.63,0.44}{{#1}}}
    \newcommand{\FloatTok}[1]{\textcolor[rgb]{0.25,0.63,0.44}{{#1}}}
    \newcommand{\CharTok}[1]{\textcolor[rgb]{0.25,0.44,0.63}{{#1}}}
    \newcommand{\StringTok}[1]{\textcolor[rgb]{0.25,0.44,0.63}{{#1}}}
    \newcommand{\CommentTok}[1]{\textcolor[rgb]{0.38,0.63,0.69}{\textit{{#1}}}}
    \newcommand{\OtherTok}[1]{\textcolor[rgb]{0.00,0.44,0.13}{{#1}}}
    \newcommand{\AlertTok}[1]{\textcolor[rgb]{1.00,0.00,0.00}{\textbf{{#1}}}}
    \newcommand{\FunctionTok}[1]{\textcolor[rgb]{0.02,0.16,0.49}{{#1}}}
    \newcommand{\RegionMarkerTok}[1]{{#1}}
    \newcommand{\ErrorTok}[1]{\textcolor[rgb]{1.00,0.00,0.00}{\textbf{{#1}}}}
    \newcommand{\NormalTok}[1]{{#1}}
    
    % Additional commands for more recent versions of Pandoc
    \newcommand{\ConstantTok}[1]{\textcolor[rgb]{0.53,0.00,0.00}{{#1}}}
    \newcommand{\SpecialCharTok}[1]{\textcolor[rgb]{0.25,0.44,0.63}{{#1}}}
    \newcommand{\VerbatimStringTok}[1]{\textcolor[rgb]{0.25,0.44,0.63}{{#1}}}
    \newcommand{\SpecialStringTok}[1]{\textcolor[rgb]{0.73,0.40,0.53}{{#1}}}
    \newcommand{\ImportTok}[1]{{#1}}
    \newcommand{\DocumentationTok}[1]{\textcolor[rgb]{0.73,0.13,0.13}{\textit{{#1}}}}
    \newcommand{\AnnotationTok}[1]{\textcolor[rgb]{0.38,0.63,0.69}{\textbf{\textit{{#1}}}}}
    \newcommand{\CommentVarTok}[1]{\textcolor[rgb]{0.38,0.63,0.69}{\textbf{\textit{{#1}}}}}
    \newcommand{\VariableTok}[1]{\textcolor[rgb]{0.10,0.09,0.49}{{#1}}}
    \newcommand{\ControlFlowTok}[1]{\textcolor[rgb]{0.00,0.44,0.13}{\textbf{{#1}}}}
    \newcommand{\OperatorTok}[1]{\textcolor[rgb]{0.40,0.40,0.40}{{#1}}}
    \newcommand{\BuiltInTok}[1]{{#1}}
    \newcommand{\ExtensionTok}[1]{{#1}}
    \newcommand{\PreprocessorTok}[1]{\textcolor[rgb]{0.74,0.48,0.00}{{#1}}}
    \newcommand{\AttributeTok}[1]{\textcolor[rgb]{0.49,0.56,0.16}{{#1}}}
    \newcommand{\InformationTok}[1]{\textcolor[rgb]{0.38,0.63,0.69}{\textbf{\textit{{#1}}}}}
    \newcommand{\WarningTok}[1]{\textcolor[rgb]{0.38,0.63,0.69}{\textbf{\textit{{#1}}}}}
    
    
    % Define a nice break command that doesn't care if a line doesn't already
    % exist.
    \def\br{\hspace*{\fill} \\* }
    % Math Jax compatibility definitions
    \def\gt{>}
    \def\lt{<}
    \let\Oldtex\TeX
    \let\Oldlatex\LaTeX
    \renewcommand{\TeX}{\textrm{\Oldtex}}
    \renewcommand{\LaTeX}{\textrm{\Oldlatex}}
    % Document parameters
    % Document title
    \title{problemset}
    
    
    
    
    
% Pygments definitions
\makeatletter
\def\PY@reset{\let\PY@it=\relax \let\PY@bf=\relax%
    \let\PY@ul=\relax \let\PY@tc=\relax%
    \let\PY@bc=\relax \let\PY@ff=\relax}
\def\PY@tok#1{\csname PY@tok@#1\endcsname}
\def\PY@toks#1+{\ifx\relax#1\empty\else%
    \PY@tok{#1}\expandafter\PY@toks\fi}
\def\PY@do#1{\PY@bc{\PY@tc{\PY@ul{%
    \PY@it{\PY@bf{\PY@ff{#1}}}}}}}
\def\PY#1#2{\PY@reset\PY@toks#1+\relax+\PY@do{#2}}

\expandafter\def\csname PY@tok@w\endcsname{\def\PY@tc##1{\textcolor[rgb]{0.73,0.73,0.73}{##1}}}
\expandafter\def\csname PY@tok@c\endcsname{\let\PY@it=\textit\def\PY@tc##1{\textcolor[rgb]{0.25,0.50,0.50}{##1}}}
\expandafter\def\csname PY@tok@cp\endcsname{\def\PY@tc##1{\textcolor[rgb]{0.74,0.48,0.00}{##1}}}
\expandafter\def\csname PY@tok@k\endcsname{\let\PY@bf=\textbf\def\PY@tc##1{\textcolor[rgb]{0.00,0.50,0.00}{##1}}}
\expandafter\def\csname PY@tok@kp\endcsname{\def\PY@tc##1{\textcolor[rgb]{0.00,0.50,0.00}{##1}}}
\expandafter\def\csname PY@tok@kt\endcsname{\def\PY@tc##1{\textcolor[rgb]{0.69,0.00,0.25}{##1}}}
\expandafter\def\csname PY@tok@o\endcsname{\def\PY@tc##1{\textcolor[rgb]{0.40,0.40,0.40}{##1}}}
\expandafter\def\csname PY@tok@ow\endcsname{\let\PY@bf=\textbf\def\PY@tc##1{\textcolor[rgb]{0.67,0.13,1.00}{##1}}}
\expandafter\def\csname PY@tok@nb\endcsname{\def\PY@tc##1{\textcolor[rgb]{0.00,0.50,0.00}{##1}}}
\expandafter\def\csname PY@tok@nf\endcsname{\def\PY@tc##1{\textcolor[rgb]{0.00,0.00,1.00}{##1}}}
\expandafter\def\csname PY@tok@nc\endcsname{\let\PY@bf=\textbf\def\PY@tc##1{\textcolor[rgb]{0.00,0.00,1.00}{##1}}}
\expandafter\def\csname PY@tok@nn\endcsname{\let\PY@bf=\textbf\def\PY@tc##1{\textcolor[rgb]{0.00,0.00,1.00}{##1}}}
\expandafter\def\csname PY@tok@ne\endcsname{\let\PY@bf=\textbf\def\PY@tc##1{\textcolor[rgb]{0.82,0.25,0.23}{##1}}}
\expandafter\def\csname PY@tok@nv\endcsname{\def\PY@tc##1{\textcolor[rgb]{0.10,0.09,0.49}{##1}}}
\expandafter\def\csname PY@tok@no\endcsname{\def\PY@tc##1{\textcolor[rgb]{0.53,0.00,0.00}{##1}}}
\expandafter\def\csname PY@tok@nl\endcsname{\def\PY@tc##1{\textcolor[rgb]{0.63,0.63,0.00}{##1}}}
\expandafter\def\csname PY@tok@ni\endcsname{\let\PY@bf=\textbf\def\PY@tc##1{\textcolor[rgb]{0.60,0.60,0.60}{##1}}}
\expandafter\def\csname PY@tok@na\endcsname{\def\PY@tc##1{\textcolor[rgb]{0.49,0.56,0.16}{##1}}}
\expandafter\def\csname PY@tok@nt\endcsname{\let\PY@bf=\textbf\def\PY@tc##1{\textcolor[rgb]{0.00,0.50,0.00}{##1}}}
\expandafter\def\csname PY@tok@nd\endcsname{\def\PY@tc##1{\textcolor[rgb]{0.67,0.13,1.00}{##1}}}
\expandafter\def\csname PY@tok@s\endcsname{\def\PY@tc##1{\textcolor[rgb]{0.73,0.13,0.13}{##1}}}
\expandafter\def\csname PY@tok@sd\endcsname{\let\PY@it=\textit\def\PY@tc##1{\textcolor[rgb]{0.73,0.13,0.13}{##1}}}
\expandafter\def\csname PY@tok@si\endcsname{\let\PY@bf=\textbf\def\PY@tc##1{\textcolor[rgb]{0.73,0.40,0.53}{##1}}}
\expandafter\def\csname PY@tok@se\endcsname{\let\PY@bf=\textbf\def\PY@tc##1{\textcolor[rgb]{0.73,0.40,0.13}{##1}}}
\expandafter\def\csname PY@tok@sr\endcsname{\def\PY@tc##1{\textcolor[rgb]{0.73,0.40,0.53}{##1}}}
\expandafter\def\csname PY@tok@ss\endcsname{\def\PY@tc##1{\textcolor[rgb]{0.10,0.09,0.49}{##1}}}
\expandafter\def\csname PY@tok@sx\endcsname{\def\PY@tc##1{\textcolor[rgb]{0.00,0.50,0.00}{##1}}}
\expandafter\def\csname PY@tok@m\endcsname{\def\PY@tc##1{\textcolor[rgb]{0.40,0.40,0.40}{##1}}}
\expandafter\def\csname PY@tok@gh\endcsname{\let\PY@bf=\textbf\def\PY@tc##1{\textcolor[rgb]{0.00,0.00,0.50}{##1}}}
\expandafter\def\csname PY@tok@gu\endcsname{\let\PY@bf=\textbf\def\PY@tc##1{\textcolor[rgb]{0.50,0.00,0.50}{##1}}}
\expandafter\def\csname PY@tok@gd\endcsname{\def\PY@tc##1{\textcolor[rgb]{0.63,0.00,0.00}{##1}}}
\expandafter\def\csname PY@tok@gi\endcsname{\def\PY@tc##1{\textcolor[rgb]{0.00,0.63,0.00}{##1}}}
\expandafter\def\csname PY@tok@gr\endcsname{\def\PY@tc##1{\textcolor[rgb]{1.00,0.00,0.00}{##1}}}
\expandafter\def\csname PY@tok@ge\endcsname{\let\PY@it=\textit}
\expandafter\def\csname PY@tok@gs\endcsname{\let\PY@bf=\textbf}
\expandafter\def\csname PY@tok@gp\endcsname{\let\PY@bf=\textbf\def\PY@tc##1{\textcolor[rgb]{0.00,0.00,0.50}{##1}}}
\expandafter\def\csname PY@tok@go\endcsname{\def\PY@tc##1{\textcolor[rgb]{0.53,0.53,0.53}{##1}}}
\expandafter\def\csname PY@tok@gt\endcsname{\def\PY@tc##1{\textcolor[rgb]{0.00,0.27,0.87}{##1}}}
\expandafter\def\csname PY@tok@err\endcsname{\def\PY@bc##1{\setlength{\fboxsep}{0pt}\fcolorbox[rgb]{1.00,0.00,0.00}{1,1,1}{\strut ##1}}}
\expandafter\def\csname PY@tok@kc\endcsname{\let\PY@bf=\textbf\def\PY@tc##1{\textcolor[rgb]{0.00,0.50,0.00}{##1}}}
\expandafter\def\csname PY@tok@kd\endcsname{\let\PY@bf=\textbf\def\PY@tc##1{\textcolor[rgb]{0.00,0.50,0.00}{##1}}}
\expandafter\def\csname PY@tok@kn\endcsname{\let\PY@bf=\textbf\def\PY@tc##1{\textcolor[rgb]{0.00,0.50,0.00}{##1}}}
\expandafter\def\csname PY@tok@kr\endcsname{\let\PY@bf=\textbf\def\PY@tc##1{\textcolor[rgb]{0.00,0.50,0.00}{##1}}}
\expandafter\def\csname PY@tok@bp\endcsname{\def\PY@tc##1{\textcolor[rgb]{0.00,0.50,0.00}{##1}}}
\expandafter\def\csname PY@tok@fm\endcsname{\def\PY@tc##1{\textcolor[rgb]{0.00,0.00,1.00}{##1}}}
\expandafter\def\csname PY@tok@vc\endcsname{\def\PY@tc##1{\textcolor[rgb]{0.10,0.09,0.49}{##1}}}
\expandafter\def\csname PY@tok@vg\endcsname{\def\PY@tc##1{\textcolor[rgb]{0.10,0.09,0.49}{##1}}}
\expandafter\def\csname PY@tok@vi\endcsname{\def\PY@tc##1{\textcolor[rgb]{0.10,0.09,0.49}{##1}}}
\expandafter\def\csname PY@tok@vm\endcsname{\def\PY@tc##1{\textcolor[rgb]{0.10,0.09,0.49}{##1}}}
\expandafter\def\csname PY@tok@sa\endcsname{\def\PY@tc##1{\textcolor[rgb]{0.73,0.13,0.13}{##1}}}
\expandafter\def\csname PY@tok@sb\endcsname{\def\PY@tc##1{\textcolor[rgb]{0.73,0.13,0.13}{##1}}}
\expandafter\def\csname PY@tok@sc\endcsname{\def\PY@tc##1{\textcolor[rgb]{0.73,0.13,0.13}{##1}}}
\expandafter\def\csname PY@tok@dl\endcsname{\def\PY@tc##1{\textcolor[rgb]{0.73,0.13,0.13}{##1}}}
\expandafter\def\csname PY@tok@s2\endcsname{\def\PY@tc##1{\textcolor[rgb]{0.73,0.13,0.13}{##1}}}
\expandafter\def\csname PY@tok@sh\endcsname{\def\PY@tc##1{\textcolor[rgb]{0.73,0.13,0.13}{##1}}}
\expandafter\def\csname PY@tok@s1\endcsname{\def\PY@tc##1{\textcolor[rgb]{0.73,0.13,0.13}{##1}}}
\expandafter\def\csname PY@tok@mb\endcsname{\def\PY@tc##1{\textcolor[rgb]{0.40,0.40,0.40}{##1}}}
\expandafter\def\csname PY@tok@mf\endcsname{\def\PY@tc##1{\textcolor[rgb]{0.40,0.40,0.40}{##1}}}
\expandafter\def\csname PY@tok@mh\endcsname{\def\PY@tc##1{\textcolor[rgb]{0.40,0.40,0.40}{##1}}}
\expandafter\def\csname PY@tok@mi\endcsname{\def\PY@tc##1{\textcolor[rgb]{0.40,0.40,0.40}{##1}}}
\expandafter\def\csname PY@tok@il\endcsname{\def\PY@tc##1{\textcolor[rgb]{0.40,0.40,0.40}{##1}}}
\expandafter\def\csname PY@tok@mo\endcsname{\def\PY@tc##1{\textcolor[rgb]{0.40,0.40,0.40}{##1}}}
\expandafter\def\csname PY@tok@ch\endcsname{\let\PY@it=\textit\def\PY@tc##1{\textcolor[rgb]{0.25,0.50,0.50}{##1}}}
\expandafter\def\csname PY@tok@cm\endcsname{\let\PY@it=\textit\def\PY@tc##1{\textcolor[rgb]{0.25,0.50,0.50}{##1}}}
\expandafter\def\csname PY@tok@cpf\endcsname{\let\PY@it=\textit\def\PY@tc##1{\textcolor[rgb]{0.25,0.50,0.50}{##1}}}
\expandafter\def\csname PY@tok@c1\endcsname{\let\PY@it=\textit\def\PY@tc##1{\textcolor[rgb]{0.25,0.50,0.50}{##1}}}
\expandafter\def\csname PY@tok@cs\endcsname{\let\PY@it=\textit\def\PY@tc##1{\textcolor[rgb]{0.25,0.50,0.50}{##1}}}

\def\PYZbs{\char`\\}
\def\PYZus{\char`\_}
\def\PYZob{\char`\{}
\def\PYZcb{\char`\}}
\def\PYZca{\char`\^}
\def\PYZam{\char`\&}
\def\PYZlt{\char`\<}
\def\PYZgt{\char`\>}
\def\PYZsh{\char`\#}
\def\PYZpc{\char`\%}
\def\PYZdl{\char`\$}
\def\PYZhy{\char`\-}
\def\PYZsq{\char`\'}
\def\PYZdq{\char`\"}
\def\PYZti{\char`\~}
% for compatibility with earlier versions
\def\PYZat{@}
\def\PYZlb{[}
\def\PYZrb{]}
\makeatother


    % For linebreaks inside Verbatim environment from package fancyvrb. 
    \makeatletter
        \newbox\Wrappedcontinuationbox 
        \newbox\Wrappedvisiblespacebox 
        \newcommand*\Wrappedvisiblespace {\textcolor{red}{\textvisiblespace}} 
        \newcommand*\Wrappedcontinuationsymbol {\textcolor{red}{\llap{\tiny$\m@th\hookrightarrow$}}} 
        \newcommand*\Wrappedcontinuationindent {3ex } 
        \newcommand*\Wrappedafterbreak {\kern\Wrappedcontinuationindent\copy\Wrappedcontinuationbox} 
        % Take advantage of the already applied Pygments mark-up to insert 
        % potential linebreaks for TeX processing. 
        %        {, <, #, %, $, ' and ": go to next line. 
        %        _, }, ^, &, >, - and ~: stay at end of broken line. 
        % Use of \textquotesingle for straight quote. 
        \newcommand*\Wrappedbreaksatspecials {% 
            \def\PYGZus{\discretionary{\char`\_}{\Wrappedafterbreak}{\char`\_}}% 
            \def\PYGZob{\discretionary{}{\Wrappedafterbreak\char`\{}{\char`\{}}% 
            \def\PYGZcb{\discretionary{\char`\}}{\Wrappedafterbreak}{\char`\}}}% 
            \def\PYGZca{\discretionary{\char`\^}{\Wrappedafterbreak}{\char`\^}}% 
            \def\PYGZam{\discretionary{\char`\&}{\Wrappedafterbreak}{\char`\&}}% 
            \def\PYGZlt{\discretionary{}{\Wrappedafterbreak\char`\<}{\char`\<}}% 
            \def\PYGZgt{\discretionary{\char`\>}{\Wrappedafterbreak}{\char`\>}}% 
            \def\PYGZsh{\discretionary{}{\Wrappedafterbreak\char`\#}{\char`\#}}% 
            \def\PYGZpc{\discretionary{}{\Wrappedafterbreak\char`\%}{\char`\%}}% 
            \def\PYGZdl{\discretionary{}{\Wrappedafterbreak\char`\$}{\char`\$}}% 
            \def\PYGZhy{\discretionary{\char`\-}{\Wrappedafterbreak}{\char`\-}}% 
            \def\PYGZsq{\discretionary{}{\Wrappedafterbreak\textquotesingle}{\textquotesingle}}% 
            \def\PYGZdq{\discretionary{}{\Wrappedafterbreak\char`\"}{\char`\"}}% 
            \def\PYGZti{\discretionary{\char`\~}{\Wrappedafterbreak}{\char`\~}}% 
        } 
        % Some characters . , ; ? ! / are not pygmentized. 
        % This macro makes them "active" and they will insert potential linebreaks 
        \newcommand*\Wrappedbreaksatpunct {% 
            \lccode`\~`\.\lowercase{\def~}{\discretionary{\hbox{\char`\.}}{\Wrappedafterbreak}{\hbox{\char`\.}}}% 
            \lccode`\~`\,\lowercase{\def~}{\discretionary{\hbox{\char`\,}}{\Wrappedafterbreak}{\hbox{\char`\,}}}% 
            \lccode`\~`\;\lowercase{\def~}{\discretionary{\hbox{\char`\;}}{\Wrappedafterbreak}{\hbox{\char`\;}}}% 
            \lccode`\~`\:\lowercase{\def~}{\discretionary{\hbox{\char`\:}}{\Wrappedafterbreak}{\hbox{\char`\:}}}% 
            \lccode`\~`\?\lowercase{\def~}{\discretionary{\hbox{\char`\?}}{\Wrappedafterbreak}{\hbox{\char`\?}}}% 
            \lccode`\~`\!\lowercase{\def~}{\discretionary{\hbox{\char`\!}}{\Wrappedafterbreak}{\hbox{\char`\!}}}% 
            \lccode`\~`\/\lowercase{\def~}{\discretionary{\hbox{\char`\/}}{\Wrappedafterbreak}{\hbox{\char`\/}}}% 
            \catcode`\.\active
            \catcode`\,\active 
            \catcode`\;\active
            \catcode`\:\active
            \catcode`\?\active
            \catcode`\!\active
            \catcode`\/\active 
            \lccode`\~`\~ 	
        }
    \makeatother

    \let\OriginalVerbatim=\Verbatim
    \makeatletter
    \renewcommand{\Verbatim}[1][1]{%
        %\parskip\z@skip
        \sbox\Wrappedcontinuationbox {\Wrappedcontinuationsymbol}%
        \sbox\Wrappedvisiblespacebox {\FV@SetupFont\Wrappedvisiblespace}%
        \def\FancyVerbFormatLine ##1{\hsize\linewidth
            \vtop{\raggedright\hyphenpenalty\z@\exhyphenpenalty\z@
                \doublehyphendemerits\z@\finalhyphendemerits\z@
                \strut ##1\strut}%
        }%
        % If the linebreak is at a space, the latter will be displayed as visible
        % space at end of first line, and a continuation symbol starts next line.
        % Stretch/shrink are however usually zero for typewriter font.
        \def\FV@Space {%
            \nobreak\hskip\z@ plus\fontdimen3\font minus\fontdimen4\font
            \discretionary{\copy\Wrappedvisiblespacebox}{\Wrappedafterbreak}
            {\kern\fontdimen2\font}%
        }%
        
        % Allow breaks at special characters using \PYG... macros.
        \Wrappedbreaksatspecials
        % Breaks at punctuation characters . , ; ? ! and / need catcode=\active 	
        \OriginalVerbatim[#1,codes*=\Wrappedbreaksatpunct]%
    }
    \makeatother

    % Exact colors from NB
    \definecolor{incolor}{HTML}{303F9F}
    \definecolor{outcolor}{HTML}{D84315}
    \definecolor{cellborder}{HTML}{CFCFCF}
    \definecolor{cellbackground}{HTML}{F7F7F7}
    
    % prompt
    \makeatletter
    \newcommand{\boxspacing}{\kern\kvtcb@left@rule\kern\kvtcb@boxsep}
    \makeatother
    \newcommand{\prompt}[4]{
        \ttfamily\llap{{\color{#2}[#3]:\hspace{3pt}#4}}\vspace{-\baselineskip}
    }
    

    
    % Prevent overflowing lines due to hard-to-break entities
    \sloppy 
    % Setup hyperref package
    \hypersetup{
      breaklinks=true,  % so long urls are correctly broken across lines
      colorlinks=true,
      urlcolor=urlcolor,
      linkcolor=linkcolor,
      citecolor=citecolor,
      }
    % Slightly bigger margins than the latex defaults
    
    \geometry{verbose,tmargin=1in,bmargin=1in,lmargin=1in,rmargin=1in}
    
    

\begin{document}
    
    \maketitle
    
    

    
    \hypertarget{econ202a-problem-set}{%
\section{ECON202A Problem Set}\label{econ202a-problem-set}}

Team members: David Johannes, Bailey Johnson, Chandni Raja, Natasha
Watkins

    \hypertarget{problem-1}{%
\subsubsection{Problem 1}\label{problem-1}}

Given \(u(c_t) = -\frac{1}{2} (\bar{c} - c_t)^2\) and \(r = \delta\),

\begin{align*}
    u'(c_t) &= \bar{c} - c_t
\end{align*}

By the Theorem (Hall, pg. 974)

\begin{align*}
    E_t u'(c_{t+1}) &= \frac{1+\delta}{1+r} u'(c_t) \\
    \Rightarrow \bar{c} - E_t c_{t+1} &= \bar{c} - c_t \\
    c_t &= E_t c_{t+1} \\
    \Rightarrow c_{t+1} &= c_t - \varepsilon_{t+1}
\end{align*}

Where \(\varepsilon_{t+1}\) is the true disturbance term, so that
\(E_t \varepsilon_{t+1} = 0\), and therefore is white noise. This is a
random walk process with no constant.

    \hypertarget{problem-2}{%
\subsubsection{Problem 2}\label{problem-2}}

The permanent-income hypothesis suggests that consumers' behaviour
incorporates expectations of their future ability to consume, i.e.,
their consumption today should reflect their expectation of future
income in all periods. Forecasting future consumption should therefore
only be a function of current consumption, as current consumption
incorporates expectations of all other variables that might affect
consumption, and so the stochastic process for income is irrelevant for
forecasting.

As consumption in the next period is only a function of current
consumption, the error term has no effect on the level of consumption,
i.e., \(E_t \varepsilon_{t+1} = 0\). Because \(c_{t+1}\) is not
explained by consumption in any period apart from \(t\), the error terms
are uncorrelated, i.e.,
\(E_t(\varepsilon_{t+1} \varepsilon_{t+\tau}) = 0\) for all
\(\tau \neq 1\). Therefore, the error term is a white noise process.

    \hypertarget{problem-3}{%
\subsubsection{Problem 3}\label{problem-3}}

When \(r=\delta\) the agent does not get any utility from investing a
unit of consumption today to consume tomorrow, since the interest he
would gain on investing is exactly the same as his preference rate for
consumption today. However, when \(r>\delta\) the real rate of interest
is higher than the subjective time preference, as such he would choose a
certain level of investment since the gains from investment outweigh his
preference for consumption today. Thus, consumption would evolve as a
random walk, just as in question 2, but with a positive drift.

Mathematically: \begin{align*}
    E_t u'(c_{t+1}) &= \frac{1+\delta}{1+r} u'(c_t) \\
    \Rightarrow \bar{c} - E_t c_{t+1} &= \frac{1+\delta}{1+r} \left(\bar{c} - c_t\right) \\
    E_t c_{t+1} &= \bar{c} \left(1-\frac{1+\delta}{1+r}\right) + \frac{1+\delta}{1+r} c_t \\
    \Rightarrow c_{t+1} &= \beta_0 + \lambda c_t - \varepsilon_{t+1}
\end{align*}

Where \(\beta_0 = \frac{r-\delta}{1+r} > 0\), and
\(\lambda = \frac{1+\delta}{1+r} > 0\)

    \hypertarget{problem-4}{%
\subsubsection{Problem 4}\label{problem-4}}

    \begin{tcolorbox}[breakable, size=fbox, boxrule=1pt, pad at break*=1mm,colback=cellbackground, colframe=cellborder]
\prompt{In}{incolor}{1}{\boxspacing}
\begin{Verbatim}[commandchars=\\\{\}]
\PY{k+kn}{import} \PY{n+nn}{pandas} \PY{k}{as} \PY{n+nn}{pd}
\PY{k+kn}{from} \PY{n+nn}{fredapi} \PY{k}{import} \PY{n}{Fred}
\PY{k+kn}{import} \PY{n+nn}{matplotlib}\PY{n+nn}{.}\PY{n+nn}{pyplot} \PY{k}{as} \PY{n+nn}{plt}
\PY{k+kn}{import} \PY{n+nn}{numpy} \PY{k}{as} \PY{n+nn}{np}
\PY{k+kn}{from} \PY{n+nn}{statsmodels}\PY{n+nn}{.}\PY{n+nn}{api} \PY{k}{import} \PY{n}{OLS}\PY{p}{,} \PY{n}{tsa}
\PY{k+kn}{from} \PY{n+nn}{numpy}\PY{n+nn}{.}\PY{n+nn}{linalg} \PY{k}{import} \PY{n}{inv}
\end{Verbatim}
\end{tcolorbox}

    \begin{tcolorbox}[breakable, size=fbox, boxrule=1pt, pad at break*=1mm,colback=cellbackground, colframe=cellborder]
\prompt{In}{incolor}{2}{\boxspacing}
\begin{Verbatim}[commandchars=\\\{\}]
\PY{n}{fred} \PY{o}{=} \PY{n}{Fred}\PY{p}{(}\PY{n}{api\PYZus{}key}\PY{o}{=}\PY{l+s+s1}{\PYZsq{}}\PY{l+s+s1}{16fc433e0cb217bb8cb94bf76b981f2f}\PY{l+s+s1}{\PYZsq{}}\PY{p}{)}
\PY{n}{c} \PY{o}{=} \PY{n}{fred}\PY{o}{.}\PY{n}{get\PYZus{}series}\PY{p}{(}\PY{l+s+s1}{\PYZsq{}}\PY{l+s+s1}{A794RX0Q048SBEA}\PY{l+s+s1}{\PYZsq{}}\PY{p}{)}  \PY{c+c1}{\PYZsh{} Real personal consumption expenditures per capita}
\end{Verbatim}
\end{tcolorbox}

    \begin{tcolorbox}[breakable, size=fbox, boxrule=1pt, pad at break*=1mm,colback=cellbackground, colframe=cellborder]
\prompt{In}{incolor}{3}{\boxspacing}
\begin{Verbatim}[commandchars=\\\{\}]
\PY{n}{c}\PY{o}{.}\PY{n}{plot}\PY{p}{(}\PY{n}{title}\PY{o}{=}\PY{l+s+s1}{\PYZsq{}}\PY{l+s+s1}{Real personal consumption expenditures per capita}\PY{l+s+s1}{\PYZsq{}}\PY{p}{)}
\PY{n}{plt}\PY{o}{.}\PY{n}{show}\PY{p}{(}\PY{p}{)}
\end{Verbatim}
\end{tcolorbox}

    \begin{center}
    \adjustimage{max size={0.9\linewidth}{0.9\paperheight}}{output_7_0.png}
    \end{center}
    { \hspace*{\fill} \\}
    
    \begin{tcolorbox}[breakable, size=fbox, boxrule=1pt, pad at break*=1mm,colback=cellbackground, colframe=cellborder]
\prompt{In}{incolor}{4}{\boxspacing}
\begin{Verbatim}[commandchars=\\\{\}]
\PY{n}{data} \PY{o}{=} \PY{n}{pd}\PY{o}{.}\PY{n}{DataFrame}\PY{p}{(}\PY{p}{)}
\PY{n}{data}\PY{p}{[}\PY{l+s+s1}{\PYZsq{}}\PY{l+s+s1}{y}\PY{l+s+s1}{\PYZsq{}}\PY{p}{]} \PY{o}{=} \PY{n}{np}\PY{o}{.}\PY{n}{log}\PY{p}{(}\PY{n}{c}\PY{p}{)}\PY{o}{.}\PY{n}{values}
\PY{n}{data}\PY{p}{[}\PY{l+s+s1}{\PYZsq{}}\PY{l+s+s1}{X}\PY{l+s+s1}{\PYZsq{}}\PY{p}{]} \PY{o}{=} \PY{n}{np}\PY{o}{.}\PY{n}{log}\PY{p}{(}\PY{n}{c}\PY{p}{)}\PY{o}{.}\PY{n}{shift}\PY{p}{(}\PY{l+m+mi}{1}\PY{p}{)}\PY{o}{.}\PY{n}{values} \PY{c+c1}{\PYZsh{} Shift to obtain consumption last period}
\PY{n}{data}\PY{p}{[}\PY{l+s+s1}{\PYZsq{}}\PY{l+s+s1}{constant}\PY{l+s+s1}{\PYZsq{}}\PY{p}{]} \PY{o}{=} \PY{l+m+mi}{1}
\PY{n}{data} \PY{o}{=} \PY{n}{data}\PY{p}{[}\PY{l+m+mi}{1}\PY{p}{:}\PY{p}{]}  \PY{c+c1}{\PYZsh{} Drop first observation}
\end{Verbatim}
\end{tcolorbox}

    \begin{tcolorbox}[breakable, size=fbox, boxrule=1pt, pad at break*=1mm,colback=cellbackground, colframe=cellborder]
\prompt{In}{incolor}{5}{\boxspacing}
\begin{Verbatim}[commandchars=\\\{\}]
\PY{n}{model} \PY{o}{=} \PY{n}{OLS}\PY{p}{(}\PY{n}{data}\PY{o}{.}\PY{n}{y}\PY{p}{,} \PY{n}{data}\PY{p}{[}\PY{p}{[}\PY{l+s+s1}{\PYZsq{}}\PY{l+s+s1}{constant}\PY{l+s+s1}{\PYZsq{}}\PY{p}{,} \PY{l+s+s1}{\PYZsq{}}\PY{l+s+s1}{X}\PY{l+s+s1}{\PYZsq{}}\PY{p}{]}\PY{p}{]}\PY{p}{)}
\PY{n}{results} \PY{o}{=} \PY{n}{model}\PY{o}{.}\PY{n}{fit}\PY{p}{(}\PY{p}{)}
\PY{n+nb}{print}\PY{p}{(}\PY{n}{results}\PY{o}{.}\PY{n}{summary}\PY{p}{(}\PY{p}{)}\PY{p}{)}
\end{Verbatim}
\end{tcolorbox}

    \begin{Verbatim}[commandchars=\\\{\}]
                            OLS Regression Results
==============================================================================
Dep. Variable:                      y   R-squared:                       1.000
Model:                            OLS   Adj. R-squared:                  1.000
Method:                 Least Squares   F-statistic:                 9.886e+05
Date:                Fri, 22 Nov 2019   Prob (F-statistic):               0.00
Time:                        13:15:39   Log-Likelihood:                 990.19
No. Observations:                 290   AIC:                            -1976.
Df Residuals:                     288   BIC:                            -1969.
Df Model:                           1
Covariance Type:            nonrobust
==============================================================================
                 coef    std err          t      P>|t|      [0.025      0.975]
------------------------------------------------------------------------------
constant       0.0148      0.010      1.486      0.138      -0.005       0.034
X              0.9990      0.001    994.293      0.000       0.997       1.001
==============================================================================
Omnibus:                       66.009   Durbin-Watson:                   1.838
Prob(Omnibus):                  0.000   Jarque-Bera (JB):              494.595
Skew:                          -0.670   Prob(JB):                    3.98e-108
Kurtosis:                       9.256   Cond. No.                         212.
==============================================================================

Warnings:
[1] Standard Errors assume that the covariance matrix of the errors is correctly
specified.
    \end{Verbatim}

    \begin{tcolorbox}[breakable, size=fbox, boxrule=1pt, pad at break*=1mm,colback=cellbackground, colframe=cellborder]
\prompt{In}{incolor}{6}{\boxspacing}
\begin{Verbatim}[commandchars=\\\{\}]
\PY{n}{fig}\PY{p}{,} \PY{n}{ax} \PY{o}{=} \PY{n}{plt}\PY{o}{.}\PY{n}{subplots}\PY{p}{(}\PY{n}{figsize}\PY{o}{=}\PY{p}{(}\PY{l+m+mi}{10}\PY{p}{,} \PY{l+m+mi}{7}\PY{p}{)}\PY{p}{)}
\PY{n}{ax}\PY{o}{.}\PY{n}{plot}\PY{p}{(}\PY{n}{data}\PY{o}{.}\PY{n}{index}\PY{p}{,} \PY{n}{results}\PY{o}{.}\PY{n}{fittedvalues}\PY{p}{,} \PY{n}{label}\PY{o}{=}\PY{l+s+s1}{\PYZsq{}}\PY{l+s+s1}{Fitted values}\PY{l+s+s1}{\PYZsq{}}\PY{p}{)}
\PY{n}{ax}\PY{o}{.}\PY{n}{plot}\PY{p}{(}\PY{n}{data}\PY{o}{.}\PY{n}{y}\PY{p}{,} \PY{n}{label}\PY{o}{=}\PY{l+s+s1}{\PYZsq{}}\PY{l+s+s1}{Actual}\PY{l+s+s1}{\PYZsq{}}\PY{p}{)}
\PY{n}{plt}\PY{o}{.}\PY{n}{legend}\PY{p}{(}\PY{p}{)}
\PY{n}{plt}\PY{o}{.}\PY{n}{show}\PY{p}{(}\PY{p}{)}
\end{Verbatim}
\end{tcolorbox}

    \begin{center}
    \adjustimage{max size={0.9\linewidth}{0.9\paperheight}}{output_10_0.png}
    \end{center}
    { \hspace*{\fill} \\}
    
    \begin{tcolorbox}[breakable, size=fbox, boxrule=1pt, pad at break*=1mm,colback=cellbackground, colframe=cellborder]
\prompt{In}{incolor}{7}{\boxspacing}
\begin{Verbatim}[commandchars=\\\{\}]
\PY{n}{plt}\PY{o}{.}\PY{n}{plot}\PY{p}{(}\PY{n}{results}\PY{o}{.}\PY{n}{resid}\PY{o}{.}\PY{n}{index}\PY{p}{,} \PY{n}{results}\PY{o}{.}\PY{n}{resid}\PY{p}{)}
\PY{n}{plt}\PY{o}{.}\PY{n}{title}\PY{p}{(}\PY{l+s+s1}{\PYZsq{}}\PY{l+s+s1}{Residual plot}\PY{l+s+s1}{\PYZsq{}}\PY{p}{)}
\PY{n}{plt}\PY{o}{.}\PY{n}{show}\PY{p}{(}\PY{p}{)}
\end{Verbatim}
\end{tcolorbox}

    \begin{center}
    \adjustimage{max size={0.9\linewidth}{0.9\paperheight}}{output_11_0.png}
    \end{center}
    { \hspace*{\fill} \\}
    
    The AR(1) process appears to be a reasonable statistical model of log
consumption. The autoregressive coefficient is 0.999, indicating that
log consumption is highly autoregressive. The residual plot shows that
something similiar to a white noise process. The \(R^2\) value indicates
current consumption is explained almost perfectly by previous period's
consumption.

    \hypertarget{problem-5}{%
\subsubsection{Problem 5}\label{problem-5}}

    The capital share of income is \begin{align*}
\frac{rk}{y} &= \frac{rk}{k^\theta} = rk^{1 - \theta} \\
r &= \frac{dy}{dk} = \theta k^{\theta - 1} \\
\Rightarrow \frac{rk}{y} &= \theta = 0.25
\end{align*}

To find \(\beta\) and \(\delta\), note that the share of investment is
\begin{align*}
\frac{i}{y} = \frac{\delta k}{y} = \frac{\delta k}{k^\theta} = \delta k^{1 - \theta} = 0.25 = \theta
\end{align*} This implies \(\delta = r = 0.05\).

From the steady state version of the Euler equation we found in problem
6, we have \begin{align*}
\frac{1}{\beta} = \theta \frac{y}{k} + 1 - \delta = \theta \frac{k^\theta}{k} + 1 - \delta = \theta \frac{1}{k^{1-\theta}} + 1 - \delta = \theta \frac{\delta}{\theta} + 1 - \delta = 1
\end{align*} This implies \(\beta = 1\).

    \begin{tcolorbox}[breakable, size=fbox, boxrule=1pt, pad at break*=1mm,colback=cellbackground, colframe=cellborder]
\prompt{In}{incolor}{8}{\boxspacing}
\begin{Verbatim}[commandchars=\\\{\}]
\PY{c+c1}{\PYZsh{} Set parameter values}
\PY{n}{ρ} \PY{o}{=} \PY{l+m+mf}{0.9}
\PY{n}{σ} \PY{o}{=} \PY{n}{np}\PY{o}{.}\PY{n}{sqrt}\PY{p}{(}\PY{l+m+mf}{0.005}\PY{p}{)}
\PY{n}{θ} \PY{o}{=} \PY{l+m+mf}{0.25}
\PY{n}{δ} \PY{o}{=} \PY{n}{r} \PY{o}{=} \PY{l+m+mf}{0.05}
\PY{n}{β} \PY{o}{=} \PY{l+m+mi}{1}
\end{Verbatim}
\end{tcolorbox}

    \hypertarget{problem-6}{%
\subsubsection{Problem 6}\label{problem-6}}

We can write this model recursively as \begin{align*}
    V(k, z) = \max_{k'} \left \{ \log (c) + \beta E[V(k', z')|z] \right \}
\end{align*} subject to \begin{align*}
    k' + c &= y + (1 - \delta) k \\
    y &= zk^\theta \\
    \log z' &= \rho \log z + \varepsilon'
\end{align*}

    The first order condition is \begin{align*}
    \frac{1}{c} = \beta E[V_1(k', z')|z]
\end{align*} And the envelope condition is \begin{align*}
    V_1(k, z) = \frac{z\theta k^{\theta-1} + 1 - \delta}{c}
\end{align*} Combining, \begin{align*}
    \frac{1}{c} = \beta E \left[ \left. \frac{1}{c'} (\theta \frac{y'}{k'} + 1 - \delta) \right|z \right]
\end{align*}

    The steady state solution is given by \begin{align*}
     1 &= \beta \left[ \theta \frac{\bar{y}}{\bar{k}} + 1 - \delta \right]\\
     \bar{y} &= \bar{c} + \delta \bar{k} \\
     \bar{y} &= \bar{k}^\theta
\end{align*} We can use these to solve for the steady values of \(y\),
\(k\) and \(c\)

Rearranging, we have \begin{align*}
\bar{k} = \left(\frac{1}{\theta} \left[\frac{1}{\beta} - (1 - \delta) \right] \right)^{\frac{1}{\theta - 1}}
\end{align*}

    \begin{tcolorbox}[breakable, size=fbox, boxrule=1pt, pad at break*=1mm,colback=cellbackground, colframe=cellborder]
\prompt{In}{incolor}{9}{\boxspacing}
\begin{Verbatim}[commandchars=\\\{\}]
\PY{n}{k\PYZus{}bar} \PY{o}{=} \PY{p}{(}\PY{p}{(}\PY{l+m+mi}{1} \PY{o}{/} \PY{n}{θ}\PY{p}{)} \PY{o}{*} \PY{p}{(}\PY{p}{(}\PY{l+m+mi}{1} \PY{o}{/} \PY{n}{β}\PY{p}{)} \PY{o}{\PYZhy{}} \PY{p}{(}\PY{l+m+mi}{1} \PY{o}{\PYZhy{}} \PY{n}{δ}\PY{p}{)}\PY{p}{)}\PY{p}{)}\PY{o}{*}\PY{o}{*}\PY{p}{(}\PY{l+m+mi}{1} \PY{o}{/} \PY{p}{(}\PY{n}{θ} \PY{o}{\PYZhy{}} \PY{l+m+mi}{1}\PY{p}{)}\PY{p}{)}
\PY{n}{y\PYZus{}bar} \PY{o}{=} \PY{n}{k\PYZus{}bar}\PY{o}{*}\PY{o}{*}\PY{n}{θ}
\PY{n}{c\PYZus{}bar} \PY{o}{=} \PY{n}{y\PYZus{}bar} \PY{o}{\PYZhy{}} \PY{n}{δ} \PY{o}{*} \PY{n}{k\PYZus{}bar}
\end{Verbatim}
\end{tcolorbox}

    The solution to the planner's problem is given by \begin{align}
    \frac{1}{c_t} &= \beta E_t \left[ \frac{1}{c_{t+1}} (\theta \frac{y_{t+1}}{k_{t+1}} + 1 - \delta) \right] \tag{1}\\
    k_{t+1} + c_t &= y_t + (1 - \delta) k_t \tag{2} \\
    y_t &= z_t k_t^\theta \tag{3}
\end{align}

    The steady state versions of these are \begin{align}
    1 &= \beta \left[ \theta \frac{y}{k} + 1 - \delta \right] \tag{4} \\
    y &= c + \delta k \tag{5} \\
    y &= k^\theta \tag{6}
\end{align}

    \textbf{Log-linearization of (1)} \begin{align*}
    \frac{1}{c}e^{-\tilde{c_t}} &= \beta E_t \left[ \frac{1}{c}e^{-\tilde{c}_{t+1}} ( \frac{y}{k} e^{\tilde{y}_{t+1} - \tilde{k}_{t+1}}  + 1 - \delta) \right] \\
    \Rightarrow
    \frac{1}{c}e^{-\tilde{c_t}} &= \beta E_t \left[ \frac{1}{c}( \frac{y}{k} e^{\tilde{y}_{t+1} - \tilde{k}_{t+1}-\tilde{c}_{t+1}}  + (1 - \delta)e^{-\tilde{c}_{t+1}}) \right]
\end{align*} By Taylor's theorem, we can approximate
\(e^{\tilde{x}} \approx 1 + \tilde{x}\), so \begin{align*}
    \frac{1}{c}(1-\tilde{c_t}) &= \beta E_t \left[ \frac{1}{c} (\frac{y}{k}(1 + \tilde{y}_{t+1} - \tilde{k}_{t+1} -\tilde{c}_{t+1}) + (1 - \delta)(1-\tilde{c}_{t+1}) \right]
\end{align*}

    Multiplying by \(c\) and using (4) , we can drop the constants
\begin{align*}
    -\tilde{c_t} &= \beta \frac{y}{k}E_t[\tilde{y}_{t+1}] - \beta \frac{y}{k} \tilde{k}_{t+1} - E_t[\tilde{c}_{t+1}]
\end{align*}

    \textbf{Log-linearization of (2)} \begin{align*}
    ke^{\tilde{k}_{t+1}} + ce^{\tilde{c}_t} &= ye^{\tilde{y}_t} + (1 - \delta) ke^{\tilde{k}_t} \\
    k(1+\tilde{k}_{t+1}) + c(1+\tilde{c}_t) &= y(1+\tilde{y}_t) + (1 - \delta) k(1+\tilde{k}_t)
\end{align*}

Using (5), we can drop the constants \begin{align*}
    k\tilde{k}_{t+1} + c\tilde{c}_t &= y\tilde{y}_t + (1 - \delta) k\tilde{k}_t
\end{align*}

    \textbf{Log-linearization of (3)} \begin{align*}
    ye^{\tilde{y}_t} &= z k^\theta e^{ \tilde{z}_t + \theta \tilde{k}_t } \\
    y(1 + \tilde{y}_t) &= z k^\theta (1 + \tilde{z}_t + \theta \tilde{k}_t )
\end{align*} Using (6) and the steady state value for \(z\), we get
\begin{align*}
     \tilde{y}_t &= \tilde{z}_t +  \theta\tilde{k}_t
\end{align*}

    \hypertarget{problem-7}{%
\subsubsection{Problem 7}\label{problem-7}}

The Markovian form of the system is \begin{align*}
    E
    \begin{bmatrix}
    \tilde{c}_{t+1} \\
    \tilde{k}_{t+1} \\
    \tilde{z}_{t+1}
    \end{bmatrix}
    &=
    \begin{bmatrix}
    1 - \beta R \frac{c}{k} & R & \beta \frac{y}{k}(\rho +  R) \\
    -\frac{c}{k} & \frac{1}{\beta} & \frac{y}{k} \\
    0 & 0 & \rho
    \end{bmatrix}
    \begin{bmatrix}
    \tilde{c}_{t} \\
    \tilde{k}_{t} \\
    \tilde{z}_{t}
    \end{bmatrix} 
    =
    M
    \begin{bmatrix}
    \tilde{c}_{t} \\
    \tilde{k}_{t} \\
    \tilde{z}_{t}
    \end{bmatrix}  
\end{align*} where \(R = (\theta - 1) \frac{y}{k}\)

This was obtained by \begin{align*}
    k\tilde{k}_{t+1}  &= y\tilde{y}_t + (1 - \delta) k\tilde{k}_t - c\tilde{c}_t \\
    &= y(\tilde{z}_t +  \theta\tilde{k}_t) + (1 - \delta) k\tilde{k}_t - c\tilde{c}_t \\
    \Rightarrow
    \tilde{k}_{t+1} &= \frac{y}{k}\tilde{z}_t + \frac{y}{k}\theta\tilde{k}_t + (1 - \delta) \tilde{k}_t - \frac{c}{k}\tilde{c}_t \\
    &= \frac{y}{k}\tilde{z}_t + (1 - \delta + \frac{y}{k} \theta )\tilde{k}_t - \frac{c}{k}\tilde{c}_t \\
    &= \frac{y}{k}\tilde{z}_t + \frac{1}{\beta} \tilde{k}_t - \frac{c}{k}\tilde{c}_t \\ \\
    E_t[\tilde{c}_{t+1}]  &= \beta \frac{y}{k}E_t[\tilde{y}_{t+1}] - \beta \frac{y}{k} \tilde{k}_{t+1} + \tilde{c_t} \\
    &= \beta \frac{y}{k}E_t[\tilde{z}_{t+1} +  \theta\tilde{k}_{t+1}] - \beta \frac{y}{k} \tilde{k}_{t+1} + \tilde{c_t}\\
    &= \beta \frac{y}{k} \rho \tilde{z}_t + \beta \frac{y}{k} \theta\tilde{k}_{t+1} - \beta \frac{y}{k} \tilde{k}_{t+1} + \tilde{c_t} \\
    &= \beta \frac{y}{k} \rho \tilde{z}_t + \beta \frac{y}{k} (\theta-1) \tilde{k}_{t+1} + \tilde{c_t} \\
    &= \beta \frac{y}{k} \rho \tilde{z}_t + \beta R (\frac{y}{k}\tilde{z}_t + \frac{1}{\beta} \tilde{k}_t - \frac{c}{k}\tilde{c}_t) + \tilde{c_t} \\
    &= \beta \frac{y}{k}(\rho +  R) \tilde{z}_t + R\tilde{k}_t + (1 - \beta R\frac{c}{k})\tilde{c}_t
\end{align*}

    \begin{tcolorbox}[breakable, size=fbox, boxrule=1pt, pad at break*=1mm,colback=cellbackground, colframe=cellborder]
\prompt{In}{incolor}{10}{\boxspacing}
\begin{Verbatim}[commandchars=\\\{\}]
\PY{n}{R} \PY{o}{=} \PY{p}{(}\PY{n}{θ} \PY{o}{\PYZhy{}} \PY{l+m+mi}{1}\PY{p}{)} \PY{o}{*} \PY{p}{(}\PY{n}{y\PYZus{}bar} \PY{o}{/} \PY{n}{k\PYZus{}bar}\PY{p}{)}

\PY{n}{M} \PY{o}{=} \PY{n}{np}\PY{o}{.}\PY{n}{array}\PY{p}{(}\PY{p}{[}\PY{p}{[}\PY{l+m+mi}{1} \PY{o}{\PYZhy{}} \PY{n}{β} \PY{o}{*} \PY{n}{R} \PY{o}{*} \PY{p}{(}\PY{n}{c\PYZus{}bar} \PY{o}{/} \PY{n}{k\PYZus{}bar}\PY{p}{)}\PY{p}{,}     \PY{n}{R}\PY{p}{,} \PY{n}{β} \PY{o}{*} \PY{p}{(}\PY{n}{y\PYZus{}bar} \PY{o}{/} \PY{n}{k\PYZus{}bar}\PY{p}{)} \PY{o}{*} \PY{p}{(}\PY{n}{ρ} \PY{o}{+} \PY{n}{R}\PY{p}{)}\PY{p}{]}\PY{p}{,}
              \PY{p}{[}           \PY{o}{\PYZhy{}}\PY{p}{(}\PY{n}{c\PYZus{}bar} \PY{o}{/} \PY{n}{k\PYZus{}bar}\PY{p}{)}\PY{p}{,} \PY{l+m+mi}{1} \PY{o}{/} \PY{n}{β}\PY{p}{,}                 \PY{n}{y\PYZus{}bar} \PY{o}{/} \PY{n}{k\PYZus{}bar}\PY{p}{]}\PY{p}{,}
              \PY{p}{[}\PY{l+m+mi}{0}\PY{p}{,}                               \PY{l+m+mi}{0}\PY{p}{,}                             \PY{n}{ρ}\PY{p}{]}\PY{p}{]}\PY{p}{)}
\end{Verbatim}
\end{tcolorbox}

    \begin{tcolorbox}[breakable, size=fbox, boxrule=1pt, pad at break*=1mm,colback=cellbackground, colframe=cellborder]
\prompt{In}{incolor}{11}{\boxspacing}
\begin{Verbatim}[commandchars=\\\{\}]
\PY{n}{λ}\PY{p}{,} \PY{n}{Γ} \PY{o}{=} \PY{n}{np}\PY{o}{.}\PY{n}{linalg}\PY{o}{.}\PY{n}{eig}\PY{p}{(}\PY{n}{M}\PY{p}{)}  \PY{c+c1}{\PYZsh{} Find eigenvalues and eigenvectors}
\end{Verbatim}
\end{tcolorbox}

    \begin{tcolorbox}[breakable, size=fbox, boxrule=1pt, pad at break*=1mm,colback=cellbackground, colframe=cellborder]
\prompt{In}{incolor}{12}{\boxspacing}
\begin{Verbatim}[commandchars=\\\{\}]
\PY{n}{sort\PYZus{}i} \PY{o}{=} \PY{n}{np}\PY{o}{.}\PY{n}{argsort}\PY{p}{(}\PY{n}{λ}\PY{p}{)}  \PY{c+c1}{\PYZsh{} Sort eigenvalues from largest to smallest}
\PY{n}{λ\PYZus{}new} \PY{o}{=} \PY{n}{λ}\PY{p}{[}\PY{n}{sort\PYZus{}i}\PY{p}{]}
\PY{n}{Γ\PYZus{}new} \PY{o}{=} \PY{n}{Γ}\PY{p}{[}\PY{p}{:}\PY{p}{,} \PY{n}{sort\PYZus{}i}\PY{p}{]}    \PY{c+c1}{\PYZsh{} Resort Γ based on sorted eigenvalues}
\PY{n+nb}{print}\PY{p}{(}\PY{n}{λ\PYZus{}new}\PY{p}{)}
\end{Verbatim}
\end{tcolorbox}

    \begin{Verbatim}[commandchars=\\\{\}]
[0.86082872 0.9        1.16167128]
    \end{Verbatim}

    In our system we have one control/jump variable (consumption) and one
unstable eigenvalue. Thus, the Blanchard-Kahn condition is satisfied and
we have a unique stationary solution to the linearized model.

    \hypertarget{problem-8}{%
\subsection{Problem 8}\label{problem-8}}

    First we will construct the \(z_t\) process

    \begin{tcolorbox}[breakable, size=fbox, boxrule=1pt, pad at break*=1mm,colback=cellbackground, colframe=cellborder]
\prompt{In}{incolor}{13}{\boxspacing}
\begin{Verbatim}[commandchars=\\\{\}]
\PY{n}{np}\PY{o}{.}\PY{n}{random}\PY{o}{.}\PY{n}{seed}\PY{p}{(}\PY{l+m+mi}{0}\PY{p}{)}                         \PY{c+c1}{\PYZsh{} Set random seed}
\PY{n}{T} \PY{o}{=} \PY{l+m+mi}{1200}                                  \PY{c+c1}{\PYZsh{} Length of time series}
\PY{n}{ε} \PY{o}{=} \PY{n}{np}\PY{o}{.}\PY{n}{random}\PY{o}{.}\PY{n}{normal}\PY{p}{(}\PY{n}{scale}\PY{o}{=}\PY{n}{σ}\PY{p}{,} \PY{n}{size}\PY{o}{=}\PY{n}{T}\PY{p}{)}     \PY{c+c1}{\PYZsh{} Draw random error terms}
\PY{n}{log\PYZus{}z} \PY{o}{=} \PY{n}{np}\PY{o}{.}\PY{n}{zeros}\PY{p}{(}\PY{n}{T}\PY{p}{)}                       \PY{c+c1}{\PYZsh{} To store z process}
\PY{n}{log\PYZus{}z}\PY{p}{[}\PY{l+m+mi}{0}\PY{p}{]} \PY{o}{=} \PY{l+m+mi}{0}

\PY{c+c1}{\PYZsh{} Construct z process}
\PY{k}{for} \PY{n}{t} \PY{o+ow}{in} \PY{n+nb}{range}\PY{p}{(}\PY{l+m+mi}{1}\PY{p}{,} \PY{n}{T}\PY{p}{)}\PY{p}{:}
    \PY{n}{log\PYZus{}z}\PY{p}{[}\PY{n}{t}\PY{p}{]} \PY{o}{=} \PY{n}{ρ} \PY{o}{*} \PY{n}{log\PYZus{}z}\PY{p}{[}\PY{n}{t}\PY{o}{\PYZhy{}}\PY{l+m+mi}{1}\PY{p}{]} \PY{o}{+} \PY{n}{ε}\PY{p}{[}\PY{n}{t}\PY{p}{]}
    
\PY{n}{log\PYZus{}z} \PY{o}{=} \PY{n}{log\PYZus{}z}\PY{p}{[}\PY{l+m+mi}{1}\PY{p}{:}\PY{p}{]}  \PY{c+c1}{\PYZsh{} Remove first observation}
\end{Verbatim}
\end{tcolorbox}

    \begin{tcolorbox}[breakable, size=fbox, boxrule=1pt, pad at break*=1mm,colback=cellbackground, colframe=cellborder]
\prompt{In}{incolor}{14}{\boxspacing}
\begin{Verbatim}[commandchars=\\\{\}]
\PY{n}{z} \PY{o}{=} \PY{n}{np}\PY{o}{.}\PY{n}{exp}\PY{p}{(}\PY{n}{log\PYZus{}z}\PY{p}{)}
\PY{n}{plt}\PY{o}{.}\PY{n}{plot}\PY{p}{(}\PY{n}{z}\PY{p}{)}
\PY{n}{plt}\PY{o}{.}\PY{n}{title}\PY{p}{(}\PY{l+s+s1}{\PYZsq{}}\PY{l+s+s1}{Stochastic process \PYZdl{}z\PYZus{}t\PYZdl{}}\PY{l+s+s1}{\PYZsq{}}\PY{p}{)}
\PY{n}{plt}\PY{o}{.}\PY{n}{show}\PY{p}{(}\PY{p}{)}
\end{Verbatim}
\end{tcolorbox}

    \begin{center}
    \adjustimage{max size={0.9\linewidth}{0.9\paperheight}}{output_34_0.png}
    \end{center}
    { \hspace*{\fill} \\}
    
    We first need to construct the auxiliary variable
\(Z_t = \Gamma^{-1} X_t\). Note that

    \begin{tcolorbox}[breakable, size=fbox, boxrule=1pt, pad at break*=1mm,colback=cellbackground, colframe=cellborder]
\prompt{In}{incolor}{15}{\boxspacing}
\begin{Verbatim}[commandchars=\\\{\}]
\PY{n}{Γ\PYZus{}inv} \PY{o}{=} \PY{n}{inv}\PY{p}{(}\PY{n}{Γ\PYZus{}new}\PY{p}{)}
\PY{n+nb}{print}\PY{p}{(}\PY{n}{Γ\PYZus{}inv}\PY{p}{)}
\end{Verbatim}
\end{tcolorbox}

    \begin{Verbatim}[commandchars=\\\{\}]
[[ 0.68015074  0.73307228 -6.34743225]
 [ 0.          0.          6.42649372]
 [ 0.73307228 -0.68015074 -0.09962616]]
    \end{Verbatim}

    We will partition the matrix based on stable and unstable eigenvalues
\begin{align*}
\Gamma^{-1} &=
\begin{bmatrix}
G_{11} & G_{12} \\
G_{21} & G_{22}
\end{bmatrix}
\end{align*}

    \begin{tcolorbox}[breakable, size=fbox, boxrule=1pt, pad at break*=1mm,colback=cellbackground, colframe=cellborder]
\prompt{In}{incolor}{16}{\boxspacing}
\begin{Verbatim}[commandchars=\\\{\}]
\PY{n}{G\PYZus{}22} \PY{o}{=} \PY{n}{Γ\PYZus{}inv}\PY{p}{[}\PY{l+m+mi}{2}\PY{p}{:}\PY{p}{,} \PY{l+m+mi}{1}\PY{p}{:}\PY{p}{]}
\PY{n}{G\PYZus{}21} \PY{o}{=} \PY{n}{Γ\PYZus{}inv}\PY{p}{[}\PY{p}{:}\PY{l+m+mi}{1}\PY{p}{,} \PY{l+m+mi}{2}\PY{p}{:}\PY{p}{]}
\end{Verbatim}
\end{tcolorbox}

    Transversality requires that \begin{align*}
Z_{2, t} &= G_{21} \tilde{c}_t + G_{22}
\begin{bmatrix}
\tilde{k}_t \\
\tilde{z}_t
\end{bmatrix}
=0
\\
\Rightarrow
\tilde{c}_t &= -G_{21}^{-1} G_{22} 
\begin{bmatrix}
\tilde{k}_t \\
\tilde{z}_t
\end{bmatrix}
\end{align*}

    \begin{tcolorbox}[breakable, size=fbox, boxrule=1pt, pad at break*=1mm,colback=cellbackground, colframe=cellborder]
\prompt{In}{incolor}{17}{\boxspacing}
\begin{Verbatim}[commandchars=\\\{\}]
\PY{c+c1}{\PYZsh{} Construct series for log deviations of output, consumption and capital}
\PY{n}{c\PYZus{}tilde} \PY{o}{=} \PY{n}{np}\PY{o}{.}\PY{n}{zeros}\PY{p}{(}\PY{n}{T}\PY{p}{)}
\PY{n}{k\PYZus{}tilde} \PY{o}{=} \PY{n}{np}\PY{o}{.}\PY{n}{zeros}\PY{p}{(}\PY{n}{T}\PY{p}{)}
\PY{n}{y\PYZus{}tilde} \PY{o}{=} \PY{n}{np}\PY{o}{.}\PY{n}{zeros}\PY{p}{(}\PY{n}{T}\PY{p}{)}

\PY{k}{for} \PY{n}{t} \PY{o+ow}{in} \PY{n+nb}{range}\PY{p}{(}\PY{n}{T}\PY{o}{\PYZhy{}}\PY{l+m+mi}{1}\PY{p}{)}\PY{p}{:}
    \PY{n}{z\PYZus{}tilde} \PY{o}{=} \PY{n}{log\PYZus{}z}\PY{p}{[}\PY{n}{t}\PY{p}{]}
    \PY{n}{x} \PY{o}{=} \PY{n}{np}\PY{o}{.}\PY{n}{array}\PY{p}{(}\PY{p}{[}\PY{n}{k\PYZus{}tilde}\PY{p}{[}\PY{n}{t}\PY{p}{]}\PY{p}{,}
                  \PY{n}{z\PYZus{}tilde}\PY{p}{]}\PY{p}{)}
    \PY{n}{c\PYZus{}tilde}\PY{p}{[}\PY{n}{t}\PY{p}{]} \PY{o}{=} \PY{o}{\PYZhy{}}\PY{n}{inv}\PY{p}{(}\PY{n}{G\PYZus{}21}\PY{p}{)} \PY{o}{@} \PY{n}{G\PYZus{}22} \PY{o}{@} \PY{n}{x}
    \PY{n}{y\PYZus{}tilde}\PY{p}{[}\PY{n}{t}\PY{p}{]} \PY{o}{=} \PY{n}{z\PYZus{}tilde} \PY{o}{+} \PY{n}{θ} \PY{o}{*} \PY{n}{k\PYZus{}tilde}\PY{p}{[}\PY{n}{t}\PY{p}{]}
    \PY{n}{k\PYZus{}tilde}\PY{p}{[}\PY{n}{t}\PY{o}{+}\PY{l+m+mi}{1}\PY{p}{]} \PY{o}{=} \PY{n}{y\PYZus{}tilde}\PY{p}{[}\PY{n}{t}\PY{p}{]} \PY{o}{\PYZhy{}} \PY{n}{c\PYZus{}tilde}\PY{p}{[}\PY{n}{t}\PY{p}{]}
\end{Verbatim}
\end{tcolorbox}

    \hypertarget{problem-9}{%
\subsection{Problem 9}\label{problem-9}}

    \begin{tcolorbox}[breakable, size=fbox, boxrule=1pt, pad at break*=1mm,colback=cellbackground, colframe=cellborder]
\prompt{In}{incolor}{18}{\boxspacing}
\begin{Verbatim}[commandchars=\\\{\}]
\PY{n}{discard\PYZus{}number} \PY{o}{=} \PY{l+m+mi}{200}  \PY{c+c1}{\PYZsh{} Remove the first x number of observations}

\PY{c+c1}{\PYZsh{} These are the log deviations from the steady state}
\PY{n}{log\PYZus{}k} \PY{o}{=} \PY{n}{k\PYZus{}tilde}\PY{p}{[}\PY{n}{discard\PYZus{}number}\PY{p}{:}\PY{p}{]}
\PY{n}{log\PYZus{}c} \PY{o}{=} \PY{n}{c\PYZus{}tilde}\PY{p}{[}\PY{n}{discard\PYZus{}number}\PY{p}{:}\PY{p}{]}
\PY{n}{log\PYZus{}y} \PY{o}{=} \PY{n}{y\PYZus{}tilde}\PY{p}{[}\PY{n}{discard\PYZus{}number}\PY{p}{:}\PY{p}{]}
\end{Verbatim}
\end{tcolorbox}

    \begin{tcolorbox}[breakable, size=fbox, boxrule=1pt, pad at break*=1mm,colback=cellbackground, colframe=cellborder]
\prompt{In}{incolor}{19}{\boxspacing}
\begin{Verbatim}[commandchars=\\\{\}]
\PY{n}{fig}\PY{p}{,} \PY{n}{ax} \PY{o}{=} \PY{n}{plt}\PY{o}{.}\PY{n}{subplots}\PY{p}{(}\PY{n}{figsize}\PY{o}{=}\PY{p}{(}\PY{l+m+mi}{10}\PY{p}{,} \PY{l+m+mi}{6}\PY{p}{)}\PY{p}{)}
\PY{n}{plt}\PY{o}{.}\PY{n}{plot}\PY{p}{(}\PY{n}{log\PYZus{}c}\PY{p}{,} \PY{n}{alpha}\PY{o}{=}\PY{l+m+mf}{0.6}\PY{p}{,} \PY{n}{label}\PY{o}{=}\PY{l+s+s1}{\PYZsq{}}\PY{l+s+s1}{Consumption}\PY{l+s+s1}{\PYZsq{}}\PY{p}{)}
\PY{n}{plt}\PY{o}{.}\PY{n}{plot}\PY{p}{(}\PY{n}{log\PYZus{}y}\PY{p}{,} \PY{n}{alpha}\PY{o}{=}\PY{l+m+mf}{0.6}\PY{p}{,} \PY{n}{label}\PY{o}{=}\PY{l+s+s1}{\PYZsq{}}\PY{l+s+s1}{Output}\PY{l+s+s1}{\PYZsq{}}\PY{p}{)}
\PY{n}{plt}\PY{o}{.}\PY{n}{xlabel}\PY{p}{(}\PY{l+s+s1}{\PYZsq{}}\PY{l+s+s1}{\PYZdl{}t\PYZdl{}}\PY{l+s+s1}{\PYZsq{}}\PY{p}{)}
\PY{n}{plt}\PY{o}{.}\PY{n}{title}\PY{p}{(}\PY{l+s+s1}{\PYZsq{}}\PY{l+s+s1}{Log deviations from steady state}\PY{l+s+s1}{\PYZsq{}}\PY{p}{)}
\PY{n}{plt}\PY{o}{.}\PY{n}{legend}\PY{p}{(}\PY{p}{)}
\PY{n}{plt}\PY{o}{.}\PY{n}{show}\PY{p}{(}\PY{p}{)}
\end{Verbatim}
\end{tcolorbox}

    \begin{center}
    \adjustimage{max size={0.9\linewidth}{0.9\paperheight}}{output_43_0.png}
    \end{center}
    { \hspace*{\fill} \\}
    
    \begin{tcolorbox}[breakable, size=fbox, boxrule=1pt, pad at break*=1mm,colback=cellbackground, colframe=cellborder]
\prompt{In}{incolor}{20}{\boxspacing}
\begin{Verbatim}[commandchars=\\\{\}]
\PY{n}{model} \PY{o}{=} \PY{n}{tsa}\PY{o}{.}\PY{n}{AR}\PY{p}{(}\PY{n}{log\PYZus{}c}\PY{p}{)}
\PY{n}{results} \PY{o}{=} \PY{n}{model}\PY{o}{.}\PY{n}{fit}\PY{p}{(}\PY{n}{maxlag}\PY{o}{=}\PY{l+m+mi}{1}\PY{p}{)}
\PY{n}{results}\PY{o}{.}\PY{n}{params}\PY{o}{.}\PY{n}{round}\PY{p}{(}\PY{l+m+mi}{3}\PY{p}{)}
\end{Verbatim}
\end{tcolorbox}

            \begin{tcolorbox}[breakable, size=fbox, boxrule=.5pt, pad at break*=1mm, opacityfill=0]
\prompt{Out}{outcolor}{20}{\boxspacing}
\begin{Verbatim}[commandchars=\\\{\}]
array([0.   , 0.962])
\end{Verbatim}
\end{tcolorbox}
        
    The value of the AR(1) parameter is approximately 0.96.

    \hypertarget{problem-10}{%
\subsection{Problem 10}\label{problem-10}}

We now have \[
\frac{rk}{y} = \frac{rk}{k^\theta} = rk^{1 - \theta} = 0.65 = \theta
\]

To find \(\delta\), note that the share of investment is \[
\frac{i}{y} = \frac{\delta k}{y} = \frac{\delta k}{k^\theta} = \delta k^{1 - \theta} = 0.25
\]

Therefore, \[
\frac{\delta k^{1 - \theta}}{r k^{1 - \theta}}
= \frac{\delta}{r} = \frac{0.25}{\theta}
\] so \[
\delta = \frac{0.25}{\theta} \cdot r
\]

From the steady state version of the euler equation, \[
\frac{1}{\beta} 
= \theta \frac{y}{k} + 1 - \delta 
= \theta \frac{k^\theta}{k} + 1 - \delta 
= \theta \frac{1}{k^{1-\theta}} + 1 - \delta 
= \theta \frac{r}{\theta} + 1 - \delta
= r + 1 - \delta
\] This implies \(\beta = \frac{1}{r + 1 - \delta}\)

    \begin{tcolorbox}[breakable, size=fbox, boxrule=1pt, pad at break*=1mm,colback=cellbackground, colframe=cellborder]
\prompt{In}{incolor}{21}{\boxspacing}
\begin{Verbatim}[commandchars=\\\{\}]
\PY{c+c1}{\PYZsh{} Find new parameters and steady state values}
\PY{n}{θ} \PY{o}{=} \PY{l+m+mf}{0.65}
\PY{n}{δ} \PY{o}{=} \PY{p}{(}\PY{l+m+mf}{0.25} \PY{o}{/} \PY{n}{θ}\PY{p}{)} \PY{o}{*} \PY{n}{r}
\PY{n}{β} \PY{o}{=} \PY{l+m+mi}{1} \PY{o}{/} \PY{p}{(}\PY{n}{r} \PY{o}{+} \PY{l+m+mi}{1} \PY{o}{\PYZhy{}} \PY{n}{δ}\PY{p}{)}
\PY{n}{k\PYZus{}bar} \PY{o}{=} \PY{p}{(}\PY{p}{(}\PY{l+m+mi}{1} \PY{o}{/} \PY{n}{θ}\PY{p}{)} \PY{o}{*} \PY{p}{(}\PY{p}{(}\PY{l+m+mi}{1} \PY{o}{/} \PY{n}{β}\PY{p}{)} \PY{o}{\PYZhy{}} \PY{p}{(}\PY{l+m+mi}{1} \PY{o}{\PYZhy{}} \PY{n}{δ}\PY{p}{)}\PY{p}{)}\PY{p}{)}\PY{o}{*}\PY{o}{*}\PY{p}{(}\PY{l+m+mi}{1} \PY{o}{/} \PY{p}{(}\PY{n}{θ} \PY{o}{\PYZhy{}} \PY{l+m+mi}{1}\PY{p}{)}\PY{p}{)}
\PY{n}{y\PYZus{}bar} \PY{o}{=} \PY{n}{k\PYZus{}bar}\PY{o}{*}\PY{o}{*}\PY{n}{θ}
\PY{n}{c\PYZus{}bar} \PY{o}{=} \PY{n}{y\PYZus{}bar} \PY{o}{\PYZhy{}} \PY{n}{δ} \PY{o}{*} \PY{n}{k\PYZus{}bar}

\PY{c+c1}{\PYZsh{} Find the Markovian matrix}
\PY{n}{R} \PY{o}{=} \PY{p}{(}\PY{n}{θ} \PY{o}{\PYZhy{}} \PY{l+m+mi}{1}\PY{p}{)} \PY{o}{*} \PY{p}{(}\PY{n}{y\PYZus{}bar} \PY{o}{/} \PY{n}{k\PYZus{}bar}\PY{p}{)}
\PY{n}{M} \PY{o}{=} \PY{n}{np}\PY{o}{.}\PY{n}{array}\PY{p}{(}\PY{p}{[}\PY{p}{[}\PY{l+m+mi}{1} \PY{o}{\PYZhy{}} \PY{n}{β} \PY{o}{*} \PY{n}{R} \PY{o}{*} \PY{p}{(}\PY{n}{c\PYZus{}bar} \PY{o}{/} \PY{n}{k\PYZus{}bar}\PY{p}{)}\PY{p}{,}     \PY{n}{R}\PY{p}{,} \PY{n}{β} \PY{o}{*} \PY{p}{(}\PY{n}{y\PYZus{}bar} \PY{o}{/} \PY{n}{k\PYZus{}bar}\PY{p}{)} \PY{o}{*} \PY{p}{(}\PY{n}{ρ} \PY{o}{+} \PY{n}{R}\PY{p}{)}\PY{p}{]}\PY{p}{,}
              \PY{p}{[}           \PY{o}{\PYZhy{}}\PY{p}{(}\PY{n}{c\PYZus{}bar} \PY{o}{/} \PY{n}{k\PYZus{}bar}\PY{p}{)}\PY{p}{,} \PY{l+m+mi}{1} \PY{o}{/} \PY{n}{β}\PY{p}{,}                 \PY{n}{y\PYZus{}bar} \PY{o}{/} \PY{n}{k\PYZus{}bar}\PY{p}{]}\PY{p}{,}
              \PY{p}{[}\PY{l+m+mi}{0}\PY{p}{,}                               \PY{l+m+mi}{0}\PY{p}{,}                             \PY{n}{ρ}\PY{p}{]}\PY{p}{]}\PY{p}{)}

\PY{c+c1}{\PYZsh{} Find eigenvalues and eigenvectors}
\PY{n}{λ}\PY{p}{,} \PY{n}{Γ} \PY{o}{=} \PY{n}{np}\PY{o}{.}\PY{n}{linalg}\PY{o}{.}\PY{n}{eig}\PY{p}{(}\PY{n}{M}\PY{p}{)}
\PY{n}{sort\PYZus{}i} \PY{o}{=} \PY{n}{np}\PY{o}{.}\PY{n}{argsort}\PY{p}{(}\PY{n}{λ}\PY{p}{)}  \PY{c+c1}{\PYZsh{} Sort eigenvalues from smallest to largest}
\PY{n}{λ\PYZus{}new} \PY{o}{=} \PY{n}{λ}\PY{p}{[}\PY{n}{sort\PYZus{}i}\PY{p}{]}
\PY{n}{λ\PYZus{}new}
\end{Verbatim}
\end{tcolorbox}

            \begin{tcolorbox}[breakable, size=fbox, boxrule=.5pt, pad at break*=1mm, opacityfill=0]
\prompt{Out}{outcolor}{21}{\boxspacing}
\begin{Verbatim}[commandchars=\\\{\}]
array([0.9       , 0.97409849, 1.05817763])
\end{Verbatim}
\end{tcolorbox}
        
    \begin{tcolorbox}[breakable, size=fbox, boxrule=1pt, pad at break*=1mm,colback=cellbackground, colframe=cellborder]
\prompt{In}{incolor}{22}{\boxspacing}
\begin{Verbatim}[commandchars=\\\{\}]
\PY{n}{Γ\PYZus{}new} \PY{o}{=} \PY{n}{Γ}\PY{p}{[}\PY{p}{:}\PY{p}{,} \PY{n}{sort\PYZus{}i}\PY{p}{]}  \PY{c+c1}{\PYZsh{} Sort eigenvectors based on eigenvalues}
\PY{n}{Γ\PYZus{}inv} \PY{o}{=} \PY{n}{inv}\PY{p}{(}\PY{n}{Γ\PYZus{}new}\PY{p}{)}
\PY{n}{Γ\PYZus{}inv}
\end{Verbatim}
\end{tcolorbox}

            \begin{tcolorbox}[breakable, size=fbox, boxrule=.5pt, pad at break*=1mm, opacityfill=0]
\prompt{Out}{outcolor}{22}{\boxspacing}
\begin{Verbatim}[commandchars=\\\{\}]
array([[ 0.        ,  0.        ,  1.67048727],
       [-0.96183323, -0.45694672, -1.32010773],
       [ 0.75966431, -0.74621284, -0.04997587]])
\end{Verbatim}
\end{tcolorbox}
        
    \begin{tcolorbox}[breakable, size=fbox, boxrule=1pt, pad at break*=1mm,colback=cellbackground, colframe=cellborder]
\prompt{In}{incolor}{23}{\boxspacing}
\begin{Verbatim}[commandchars=\\\{\}]
\PY{c+c1}{\PYZsh{} Partition the matrix based on stable and unstable eigenvalues}
\PY{n}{G\PYZus{}22} \PY{o}{=} \PY{n}{Γ\PYZus{}inv}\PY{p}{[}\PY{l+m+mi}{2}\PY{p}{:}\PY{p}{,} \PY{l+m+mi}{1}\PY{p}{:}\PY{p}{]}
\PY{n}{G\PYZus{}21} \PY{o}{=} \PY{n}{Γ\PYZus{}inv}\PY{p}{[}\PY{p}{:}\PY{l+m+mi}{1}\PY{p}{,} \PY{l+m+mi}{2}\PY{p}{:}\PY{p}{]}
\end{Verbatim}
\end{tcolorbox}

    \begin{tcolorbox}[breakable, size=fbox, boxrule=1pt, pad at break*=1mm,colback=cellbackground, colframe=cellborder]
\prompt{In}{incolor}{24}{\boxspacing}
\begin{Verbatim}[commandchars=\\\{\}]
\PY{c+c1}{\PYZsh{} Construct series for log deviations of output, consumption and capital}
\PY{n}{c\PYZus{}tilde} \PY{o}{=} \PY{n}{np}\PY{o}{.}\PY{n}{zeros}\PY{p}{(}\PY{n}{T}\PY{p}{)}
\PY{n}{k\PYZus{}tilde} \PY{o}{=} \PY{n}{np}\PY{o}{.}\PY{n}{zeros}\PY{p}{(}\PY{n}{T}\PY{p}{)}
\PY{n}{y\PYZus{}tilde} \PY{o}{=} \PY{n}{np}\PY{o}{.}\PY{n}{zeros}\PY{p}{(}\PY{n}{T}\PY{p}{)}

\PY{k}{for} \PY{n}{t} \PY{o+ow}{in} \PY{n+nb}{range}\PY{p}{(}\PY{n}{T}\PY{o}{\PYZhy{}}\PY{l+m+mi}{1}\PY{p}{)}\PY{p}{:}
    \PY{n}{z\PYZus{}tilde} \PY{o}{=} \PY{n}{log\PYZus{}z}\PY{p}{[}\PY{n}{t}\PY{p}{]}
    \PY{n}{x} \PY{o}{=} \PY{n}{np}\PY{o}{.}\PY{n}{array}\PY{p}{(}\PY{p}{[}\PY{n}{k\PYZus{}tilde}\PY{p}{[}\PY{n}{t}\PY{p}{]}\PY{p}{,}
                  \PY{n}{z\PYZus{}tilde}\PY{p}{]}\PY{p}{)}
    \PY{n}{c\PYZus{}tilde}\PY{p}{[}\PY{n}{t}\PY{p}{]} \PY{o}{=} \PY{o}{\PYZhy{}}\PY{n}{inv}\PY{p}{(}\PY{n}{G\PYZus{}21}\PY{p}{)} \PY{o}{@} \PY{n}{G\PYZus{}22} \PY{o}{@} \PY{n}{x}
    \PY{n}{y\PYZus{}tilde}\PY{p}{[}\PY{n}{t}\PY{p}{]} \PY{o}{=} \PY{n}{z\PYZus{}tilde} \PY{o}{+} \PY{n}{θ} \PY{o}{*} \PY{n}{k\PYZus{}tilde}\PY{p}{[}\PY{n}{t}\PY{p}{]}
    \PY{n}{k\PYZus{}tilde}\PY{p}{[}\PY{n}{t}\PY{o}{+}\PY{l+m+mi}{1}\PY{p}{]} \PY{o}{=} \PY{n}{y\PYZus{}tilde}\PY{p}{[}\PY{n}{t}\PY{p}{]} \PY{o}{\PYZhy{}} \PY{n}{c\PYZus{}tilde}\PY{p}{[}\PY{n}{t}\PY{p}{]}
\end{Verbatim}
\end{tcolorbox}

    \begin{tcolorbox}[breakable, size=fbox, boxrule=1pt, pad at break*=1mm,colback=cellbackground, colframe=cellborder]
\prompt{In}{incolor}{25}{\boxspacing}
\begin{Verbatim}[commandchars=\\\{\}]
\PY{c+c1}{\PYZsh{} These are the log deviations from the steady state}
\PY{n}{log\PYZus{}k} \PY{o}{=} \PY{n}{k\PYZus{}tilde}\PY{p}{[}\PY{n}{discard\PYZus{}number}\PY{p}{:}\PY{p}{]}
\PY{n}{log\PYZus{}c} \PY{o}{=} \PY{n}{c\PYZus{}tilde}\PY{p}{[}\PY{n}{discard\PYZus{}number}\PY{p}{:}\PY{p}{]}
\PY{n}{log\PYZus{}y} \PY{o}{=} \PY{n}{y\PYZus{}tilde}\PY{p}{[}\PY{n}{discard\PYZus{}number}\PY{p}{:}\PY{p}{]}
\end{Verbatim}
\end{tcolorbox}

    \begin{tcolorbox}[breakable, size=fbox, boxrule=1pt, pad at break*=1mm,colback=cellbackground, colframe=cellborder]
\prompt{In}{incolor}{26}{\boxspacing}
\begin{Verbatim}[commandchars=\\\{\}]
\PY{n}{fig}\PY{p}{,} \PY{n}{ax} \PY{o}{=} \PY{n}{plt}\PY{o}{.}\PY{n}{subplots}\PY{p}{(}\PY{n}{figsize}\PY{o}{=}\PY{p}{(}\PY{l+m+mi}{10}\PY{p}{,} \PY{l+m+mi}{6}\PY{p}{)}\PY{p}{)}
\PY{n}{plt}\PY{o}{.}\PY{n}{plot}\PY{p}{(}\PY{n}{log\PYZus{}c}\PY{p}{,} \PY{n}{alpha}\PY{o}{=}\PY{l+m+mf}{0.6}\PY{p}{,} \PY{n}{label}\PY{o}{=}\PY{l+s+s1}{\PYZsq{}}\PY{l+s+s1}{Consumption}\PY{l+s+s1}{\PYZsq{}}\PY{p}{)}
\PY{n}{plt}\PY{o}{.}\PY{n}{plot}\PY{p}{(}\PY{n}{log\PYZus{}y}\PY{p}{,} \PY{n}{alpha}\PY{o}{=}\PY{l+m+mf}{0.6}\PY{p}{,} \PY{n}{label}\PY{o}{=}\PY{l+s+s1}{\PYZsq{}}\PY{l+s+s1}{Output}\PY{l+s+s1}{\PYZsq{}}\PY{p}{)}
\PY{n}{plt}\PY{o}{.}\PY{n}{xlabel}\PY{p}{(}\PY{l+s+s1}{\PYZsq{}}\PY{l+s+s1}{\PYZdl{}t\PYZdl{}}\PY{l+s+s1}{\PYZsq{}}\PY{p}{)}
\PY{n}{plt}\PY{o}{.}\PY{n}{title}\PY{p}{(}\PY{l+s+s1}{\PYZsq{}}\PY{l+s+s1}{Log deviations from steady state}\PY{l+s+s1}{\PYZsq{}}\PY{p}{)}
\PY{n}{plt}\PY{o}{.}\PY{n}{legend}\PY{p}{(}\PY{p}{)}
\PY{n}{plt}\PY{o}{.}\PY{n}{show}\PY{p}{(}\PY{p}{)}
\end{Verbatim}
\end{tcolorbox}

    \begin{center}
    \adjustimage{max size={0.9\linewidth}{0.9\paperheight}}{output_52_0.png}
    \end{center}
    { \hspace*{\fill} \\}
    
    \begin{tcolorbox}[breakable, size=fbox, boxrule=1pt, pad at break*=1mm,colback=cellbackground, colframe=cellborder]
\prompt{In}{incolor}{27}{\boxspacing}
\begin{Verbatim}[commandchars=\\\{\}]
\PY{c+c1}{\PYZsh{} Fit the AR(1) model}
\PY{n}{model} \PY{o}{=} \PY{n}{tsa}\PY{o}{.}\PY{n}{AR}\PY{p}{(}\PY{n}{log\PYZus{}c}\PY{p}{)}
\PY{n}{results} \PY{o}{=} \PY{n}{model}\PY{o}{.}\PY{n}{fit}\PY{p}{(}\PY{n}{maxlag}\PY{o}{=}\PY{l+m+mi}{1}\PY{p}{)}
\PY{n}{results}\PY{o}{.}\PY{n}{params}
\end{Verbatim}
\end{tcolorbox}

            \begin{tcolorbox}[breakable, size=fbox, boxrule=.5pt, pad at break*=1mm, opacityfill=0]
\prompt{Out}{outcolor}{27}{\boxspacing}
\begin{Verbatim}[commandchars=\\\{\}]
array([-0.00097278,  0.94267595])
\end{Verbatim}
\end{tcolorbox}
        
    The earlier model is closer to a random walk for consumption (as
indicated by an AR(1) parameter of 0.96 versus 0.94 for this model).

The economic reason for this is that as \(\theta \to 1\), the marginal
product of capital, \(\frac{dy}{dk} = \theta k^{\theta-1}\), becomes
more insensitive to changes in capital stock. Therefore, their is less
incentive for people to adjust their consumption according to shocks to
output, and thus log consumption exhibits higher autocorrelation. We
also see this in the two plots of log deviations of output and
consumption. The first model shows much smoother (and more
autoregressive) consumption relative to changes in output.

    \begin{tcolorbox}[breakable, size=fbox, boxrule=1pt, pad at break*=1mm,colback=cellbackground, colframe=cellborder]
\prompt{In}{incolor}{ }{\boxspacing}
\begin{Verbatim}[commandchars=\\\{\}]

\end{Verbatim}
\end{tcolorbox}


    % Add a bibliography block to the postdoc
    
    
    
\end{document}
